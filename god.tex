\textbf{HURBANCHAL UNIVERSITY}

\textbf{Hello}

\textbf{hi}

\textbf{\hfill\break
}\includegraphics[width=0.94931in,height=0.9375in,alt={http://khec.edu.np/images/logo.png}]{/media/imagea.png}

\textbf{DEPARTMENT OF COMPUTER ENGINEERING}

\textbf{KHWOPA ENGINEERING COLLEGE\\
LIBALI-08, BHAKTAPUR}

\textbf{A MID-TERM REPORT}

\textbf{ON}

\textbf{Literature Efficient Architecture Folio (LEAF)}

Project work submitted in partial fulfillment of requirements for the
award of the degree of Bachelor of Engineering in Computer Engineering
(Third Semester)

\textbf{SUBMITTED BY}

Jayadev Tripathi (780314)

Sahana Shrestha (78036)

Sanjib Kasti (780339)

Upendra Shahi (780347)

\textbf{UNDER THE SUPERVISION OF}

Er. Avijit Karn

06 September 2023

\textbf{ACKNOWLEDGEMENT}

We would like to express our heartily thanks to our \textbf{Purbanchal
University} for including Project I as part of the curriculum and thank
our project supervisor \textbf{Er. Avjit Karn} who has supervised and
motivated us to fulfill this project before the timeline.~

~We also extend our deepest gratitude to our \textbf{Khwopa Engineering
College} and Department of Computer Engineering (\textbf{HOD. Er. Bikash
Chawal} and all the faculty members) for providing us with a wonderful
environment, excellent guidance, and supervision.~~

~

\textbf{With Regards,~}~

Jayadev Tripathi~~

Sahana Shrestha~~

Sanjib Kasti~

Upendra Shahi~

\textbf{\hfill\break
}

\textbf{ABSTRACT}

Our project LEAF (Literature Efficient Architecture Folio) is an
architecture for the manipulation of literature-based contents and
creations. It is an efficient platform for all literature
enthusiasts.~The system is based on the localized server (i.e. data are
stored only in the local system or user's system). Supporting multiple
users to engage in a system and have their interaction, the system
provides secured sessions. Also providing limiting functionality to the
guest users, it is totally concerned with promoting enthusiasm in
literature and helping in organizing users' creations enabling them to
create their project and collaborate by adding the members who can
access their own credentials in the same system.~~

This console-based application will also allow the user to view their
analytics and engagements, read the loaded contents, and bookmark the
preferred and impressive ones. Tying with the fluent, organized, and
user-friendly experience; Project LEAF will be part of taking your
literature experience to new heights~

~

\emph{\textbf{Keywords}}: - Analytics, LEAF, Literature, Articles,
LeafHandle~

\textbf{List of Figures}

\hyperref[_Toc144842058]{Figure 4.2.1: FlowChart 5}

\hyperref[_Toc144842663]{Figure 4.3.1: UML Activity Diagram 6}

\hyperref[_Toc144842664]{Figure 5.1.1: Home Page 7}

\hyperref[_Toc144842059]{Figure 5.1.2:Login Selection 7}

\hyperref[_Toc144842666]{Figure 5.1.3: Register Page 8}

\hyperref[_Toc144842667]{Figure 5.1.4: Read Section {[}HOME{]} 8}

\hyperref[_Toc144842668]{Figure 5.1.5: Read Section {[}Literature{]} 8}

\hyperref[_Toc144842669]{Figure 5.1.6: Read Section {[}Poem{]} 9}

\section{TABLE OF CONTENTS}\label{table-of-contents}

\hyperref[introduction]{INTRODUCTION \hyperref[introduction]{1}}

\hyperref[background]{1.1 BACKGROUND \hyperref[background]{1}}

\hyperref[problem-statement]{1.2 PROBLEM STATEMENT
\hyperref[problem-statement]{1}}

\hyperref[objectives]{1.3 OBJECTIVES \hyperref[objectives]{1}}

\hyperref[feature-and-functionalities-of-leaf-application]{1.4 FEATURE
AND FUNCTIONALITIES OF LEAF APPLICATION
\hyperref[feature-and-functionalities-of-leaf-application]{2}}

\hyperref[significance-and-scope]{1.5 SIGNIFICANCE AND SCOPE
\hyperref[significance-and-scope]{2}}

\hyperref[chapter-2]{CHAPTER 2 \hyperref[chapter-2]{3}}

\hyperref[literature-review]{LITERATURE REVIEW
\hyperref[literature-review]{3}}

\hyperref[chapter-3]{CHAPTER 3 \hyperref[chapter-3]{4}}

\hyperref[project-management]{PROJECT MANAGEMENT
\hyperref[project-management]{4}}

\hyperref[team-members]{3.1 TEAM MEMBERS \hyperref[team-members]{4}}

\hyperref[work-breakdown-planning]{3.2 WORK BREAKDOWN PLANNING
\hyperref[work-breakdown-planning]{4}}

\hyperref[feasibility-study]{3.3 FEASIBILITY STUDY
\hyperref[feasibility-study]{4}}

\hyperref[chapter-4]{CHAPTER 4 \hyperref[chapter-4]{5}}

\hyperref[methodology]{METHODOLOGY \hyperref[methodology]{5}}

\hyperref[algorithm]{4.1 ALGORITHM \hyperref[algorithm]{5}}

\hyperref[flowchart]{4.2 FLOWCHART \hyperref[flowchart]{6}}

\hyperref[uml-activity-diagram]{4.3 UML ACTIVITY DIAGRAM
\hyperref[uml-activity-diagram]{7}}

\hyperref[tools-and-platforms]{4.4 TOOLS AND PLATFORMS
\hyperref[tools-and-platforms]{7}}

\hyperref[chapter-5]{CHAPTER 5 \hyperref[chapter-5]{8}}

\hyperref[project-work-status]{PROJECT WORK STATUS
\hyperref[project-work-status]{8}}

\hyperref[work-done]{5.1 WORK DONE \hyperref[work-done]{8}}

\hyperref[work-remaining]{5.2 WORK REMAINING
\hyperref[work-remaining]{10}}

\hyperref[chapter-6]{CHAPTER 6 \hyperref[chapter-6]{11}}

\hyperref[references]{REFERENCES \hyperref[references]{11}}

\textbf{CHATPTER 1}

\section{INTRODUCTION}\label{introduction}

\subsection{BACKGROUND}\label{background}

In today's digital world, ever field has been randomly digitized. To
cope with the efficient digitization in the field of literature LEAF has
been developed with prime motive to enrich the user experience with the
efficient architecture and user-friendly interface. This project aims to
ease users with exciting features and promotes their enthusiasm in
literature.~

The system allows the user to register into the system and also allows
to create the new user. This facilitates the secured sessions which
includes logging out user automatically after certain course of
inactivity. It remembers the last logged user too. It even allows the
user to interact with the system without registering or logging in; but
with the limited functionalities. To gain the complete system user
registration and logging in is mandatory,~~

After you login to the system, there is option to choose between
general, professional and the management mode. Management mode includes
logging you out, changing the user credentials and managing your
collaborators. As, every author or literature enthusiasts want their
literature identity by their preferred nickname, this system facilitates
this too under this section.~~

Included in the professional mode, you can carry out any specific
literature projects by adding collaborators, setting the remainders and
tracking your each session\textquotesingle s progresses. You can even
give the relevant system feedbacks.~~

Inside into the general mode, you can use the available templates, read
and add the contents, keep your general notes, have conversation with
the collaborators, bookmark the preferred articles and view the logs and
the project description.~~

Beside these the system provides search functionalities, tags management
too.~

Once you terminate the program, you are automatically logged out of the
system and requires to enter the password again to enter into the
system.~~~~~

\subsection{\texorpdfstring{PROBLEM STATEMENT
}{PROBLEM STATEMENT }}\label{problem-statement}

In this digital era, everyone desires to organize their notes in the
regular basis. They want to experience their literature experience in a
single platform having multiple functionalities and simple and efficient
interface. They want to keep records, add teams, have conversation,
manage tags, have search functionality and track progress too. And this
is mandatory if one wants to move ahead in any respective field.~~

\subsection{OBJECTIVES}\label{objectives}

The main objective of project LEAF is to develop a user friendly and
multi-functionality literature repository.~~

\subsection{FEATURE AND FUNCTIONALITIES OF LEAF
APPLICATION}\label{feature-and-functionalities-of-leaf-application}

\begin{itemize}
\item
  User Registration and Login~
\item
  Create, Edit, and Delete Literature Entries~
\item
  Categorization, Tagging and Bookmarking~
\item
  Search Functionality~
\item
  Formatting and Styling~
\end{itemize}

\begin{itemize}
\item
  Import and Export~
\item
  Random Post Display~
\item
  User management~
\item
  Security~
\item
  Graphical User Interface {[}GUI{]}~
\end{itemize}

\begin{itemize}
\item
  Error handling~
\item
  Documentation and Help~
\end{itemize}

\subsection{SIGNIFICANCE AND SCOPE}\label{significance-and-scope}

This project LEAF will be outstanding platform to explore the literature
field. This will be useful for one seeking for the tool to enable them
to record their creation and gather as a project. This will entirely
give outstanding experience of the literature field and helps the
individuals to explore their creativity, record it in organized manner,
and view the progress track. This also enables to upload ones favorite
contents and read it during the leisure.~~

\section{CHAPTER 2}\label{chapter-2}

\section{LITERATURE REVIEW}\label{literature-review}

In this era of advanced technologies, the world is caught up in a single
computer or digital device. Every moment and all human activities are
associated with the digital platform. Aside, literature is the magic
that can overcome sorrows, pain, and loneliness and motivates in every
failure and further encourage to achieve the goal. In the digital era,
the development of applications provides an amazing platform for
organizing and creating literature.~

The application like memo in the early keypad phones to the evolved
application like: Google Keep, Evernote, Microsoft OneNote, To do List,
and soon clarifies the importance of the habit of note takings. On the
other side, Evernote, Instapaper, Pocket, Flipboard, Read wise, etc.
applications are popular for accessing and reading the literature
contents.~

``Notes are stored in virtual "notebooks" and can be tagged, annotated,
edited, searched, and exported.'' Evernote

Neil Gaiman: "Software development is an ally to authors, empowering us
to explore innovative narrative structures and collaborate with readers
in unprecedented ways. The digital realm offers endless possibilities to
enhance and expand the reader\textquotesingle s experience."~

Salman Rushdie: "Technology has reshaped the literary landscape,
enabling us to transcend traditional boundaries. Through software
development, we can create multi-dimensional narratives and amplify
diverse voices, fostering a truly global literary community."~

``You can create, edit, and share notes with Google Keep.''

From the above statements from prominent authors, as a user, every
individual wants a well-organized application that not only allows one
to create notes and general articles or just provides written articles
to look over, but they also want wide range of features that enhance the
overall literature experience. Google Keep is concerned only with
keeping, editing, and sharing notes while Evernote facilitates reading
articles more. However, they lack allowing user to create their typical
literature projects, classifying the literature genre, set the nickname,
track their analytics, add the collaborator securely, manage tags and
search for the prominent literature figures in the same platform. As the
one to address these issues and promote the journey through the magical
world of literature, the project ``LEAF'' has been developed.~~

\section{CHAPTER 3}\label{chapter-3}

\section{PROJECT MANAGEMENT~}\label{project-management}

\textbf{~}~

\begin{enumerate}
\def\labelenumi{\arabic{enumi}.}
\item
\item
\end{enumerate}

\subsection{TEAM MEMBERS}\label{team-members}

The project is carried out by the contribution of the following four
team members.~~

\begin{itemize}
\item
  Jayadev Tripathi {[}780314{]}~
\item
  Sahana Shrestha {[}780336{]}~
\item
  Sanjib Kasti {[}780339{]}~
\end{itemize}

\begin{itemize}
\item
  Upendra Shahi {[}780347{]}~
\end{itemize}

\subsection{WORK BREAKDOWN PLANNING~}\label{work-breakdown-planning}

The table shows the work breakdown planning of the project duration.

\begin{longtable}[]{@{}
  >{\centering\arraybackslash}p{(\linewidth - 16\tabcolsep) * \real{0.0643}}
  >{\centering\arraybackslash}p{(\linewidth - 16\tabcolsep) * \real{0.3278}}
  >{\centering\arraybackslash}p{(\linewidth - 16\tabcolsep) * \real{0.1191}}
  >{\centering\arraybackslash}p{(\linewidth - 16\tabcolsep) * \real{0.0815}}
  >{\centering\arraybackslash}p{(\linewidth - 16\tabcolsep) * \real{0.0815}}
  >{\centering\arraybackslash}p{(\linewidth - 16\tabcolsep) * \real{0.0815}}
  >{\centering\arraybackslash}p{(\linewidth - 16\tabcolsep) * \real{0.0815}}
  >{\centering\arraybackslash}p{(\linewidth - 16\tabcolsep) * \real{0.0815}}
  >{\centering\arraybackslash}p{(\linewidth - 16\tabcolsep) * \real{0.0815}}@{}}
\toprule\noalign{}
\begin{minipage}[b]{\linewidth}\centering
S.N.~
\end{minipage} & \begin{minipage}[b]{\linewidth}\raggedleft
~~~~~~~~~~~~~~~~~~~~~~~~~~~~~~~ Week~

~

Task Description~
\end{minipage} & \begin{minipage}[b]{\linewidth}\centering
Duration~
\end{minipage} & \begin{minipage}[b]{\linewidth}\centering
1\textsuperscript{st} Week~
\end{minipage} & \begin{minipage}[b]{\linewidth}\centering
2\textsuperscript{nd} Week~
\end{minipage} & \begin{minipage}[b]{\linewidth}\centering
3\textsuperscript{rd} Week~
\end{minipage} & \begin{minipage}[b]{\linewidth}\centering
4\textsuperscript{th} Week~
\end{minipage} & \begin{minipage}[b]{\linewidth}\centering
5\textsuperscript{th} Week~
\end{minipage} & \begin{minipage}[b]{\linewidth}\centering
6\textsuperscript{th} Week~
\end{minipage} \\
\midrule\noalign{}
\endhead
\bottomrule\noalign{}
\endlastfoot
1~ & Problem Identification~ & 3~ & ~ & ~ & ~ & ~ & ~ & ~ \\
2~ & Analysis~ & 5~ & ~ & ~ & ~ & ~ & ~ & ~ \\
3~ & Design~ & 9~ & ~ & ~ & ~ & ~ & ~ & ~ \\
4~ & Coding~ & 22~ & ~ & ~ & ~ & ~ & ~ & ~ \\
5~ & Implementing and Testing~ & 10~ & ~ & ~ & ~ & ~ & ~ & ~ \\
6~ & Documentation~ & 27~ & ~ & ~ & ~ & ~ & ~ & ~ \\
\end{longtable}

\subsection{FEASIBILITY STUDY}\label{feasibility-study}

­

It is the aspect that analyze, studies and predict the overall success
and potential of the project under several aspects. It is based on
extensive research and investigation regarding the project scenario and
conceptual models.~~

\begin{itemize}
\item
\item
\item
\item
  \textbf{Economic Feasibility}: The project does not requires any high
  financial support as it will be performed in free platform. Further,
  no any financial needs till its completion.~~\textbf{Operation
  Feasibility:} This application will run smoothly in minimum hardware
  specifications and does not needs high
  requirements.~~\textbf{Technical Feasibility:} A normal literate
  people who can understand general English Language can easily handle
  the operation. In case of difficulty, the logs and manuals in the
  program will clarify it.~~\textbf{Schedule Feasibility:} The expected
  duration for the completion of this project is around 3 months. The
  project will be completed in the allotted time frame.~~
\end{itemize}

\section{CHAPTER 4}\label{chapter-4}

\section{METHODOLOGY}\label{methodology}

\subsection{4.1 ALGORITHM}\label{algorithm}

\begin{enumerate}
\def\labelenumi{\arabic{enumi}.}
\item
  Start:
\item
  Display introduction:~Display a welcome message and introduce the
  application to the user.
\item
  Ask user to login:~Ask the user if they would like to login or sign up
  for an account or Guest mode.
\item
  If the user chooses to login:

  \begin{enumerate}
  \def\labelenumii{\arabic{enumii}.}
  \item
    Prompt the user to enter their username and password.
  \item
    Verify the user\textquotesingle s credentials.
  \item
    If the user\textquotesingle s credentials are valid, log them in and
    display the main menu.
  \item
    If the user\textquotesingle s credentials are invalid, display an
    error message and ask them to try again.
  \end{enumerate}
\item
  If the user chooses to sign up:

  \begin{enumerate}
  \def\labelenumii{\arabic{enumii}.}
  \item
    Prompt the user to enter their name, and password.
  \item
    Create a new account for the user.
  \item
    Log the user in and display the main menu.
  \end{enumerate}
\item
  If user enters guest mode:

  \begin{enumerate}
  \def\labelenumii{\arabic{enumii}.}
  \item
    Display the main menu.
  \item
    Access to Read module only.
  \item
    If tries to access other, request login/sign up and redirect.
  \end{enumerate}
\item
  Display main menu:~Display the following list of options to the user:

  \begin{enumerate}
  \def\labelenumii{\arabic{enumii}.}
  \item
    Read literature
  \item
    Collaborate with other users
  \item
    Keep notes
  \item
    Create new content
  \item
    Change settings
  \end{enumerate}
\end{enumerate}

\begin{enumerate}
\def\labelenumi{\arabic{enumi}.}
\item
  Allow the user to select an option:~Allow the user to select one of
  the options from the main menu.
\item
  Perform the selected action:~Based on the option selected by the user,
  perform the corresponding action. For example, if the user selects
  "Read literature", display a list of all available literature.
\item
  Repeat steps 7-9 until the user selects "Exit":~Continue to display
  the main menu and allow the user to select an option until they select
  "Exit".
\item
  Exit:~Close the application.
\end{enumerate}

\subsection[4.2
FLOWCHART]{\texorpdfstring{\protect\includegraphics[width=7.26389in,height=7.05208in]{media/image2.png}4.2
FLOWCHART}{4.2 FLOWCHART}}\label{flowchart}

\protect\phantomsection\label{_Toc144842058}{}\textbf{Figure 4.2.1:
Flowchart}

\begin{enumerate}
\def\labelenumi{\arabic{enumi}.}
\setcounter{enumi}{4}
\tightlist
\item
\end{enumerate}

\subsection{4.3 UML ACTIVITY DIAGRAM}\label{uml-activity-diagram}

\protect\phantomsection\label{_Toc144842663}{}\includegraphics[width=6.82257in,height=6.71675in]{media/image3.png}

\textbf{Figure 4.3.1: Activity Diagram}

\subsection{\texorpdfstring{4.4 TOOLS AND PLATFORMS
}{4.4 TOOLS AND PLATFORMS }}\label{tools-and-platforms}

\textbf{Tools:} Code Blocks

\textbf{Platform:} Windows

\section{CHAPTER 5}\label{chapter-5}

\section{PROJECT WORK STATUS}\label{project-work-status}

\subsection{\texorpdfstring{5.1 WORK DONE
}{5.1 WORK DONE }}\label{work-done}

In this phase, we have worked on the basic application framework and
organized the data flow and arranged the file management. We jointly
worked to accomplish the project by dividing it into several modules. In
this phase, we have completed:

\begin{enumerate}
\def\labelenumi{\arabic{enumi}.}
\item
  Login Module
\item
  Read Section Module
\item
  Home Page Section
\item
  Basic Graphics Designs
\end{enumerate}

\begin{figure}
\centering
\includegraphics[width=4.98431in,height=2.8025in]{media/image4.png}
\caption{\protect\phantomsection\label{_Toc144842664}{}\textbf{Figure
5.1.1: Home Page}}
\end{figure}

\begin{figure}
\centering
\includegraphics[width=5.0844in,height=2.84019in]{media/image5.png}
\caption{\protect\phantomsection\label{_Toc144842059}{}\textbf{Figure
2.1.2: Login Selection}}
\end{figure}

\begin{figure}
\centering
\includegraphics[width=4.85962in,height=2.73993in]{media/image6.png}
\caption{\protect\phantomsection\label{_Toc144842666}{}\textbf{Figure
5.1.3: Register Page (Proposed Design)}}
\end{figure}

\begin{figure}
\centering
\includegraphics[width=4.91012in,height=2.75698in]{media/image7.png}
\caption{\protect\phantomsection\label{_Toc144842667}{}\textbf{Figure
5.1.4: Read Section {[}HOME{]}}}
\end{figure}

\begin{figure}
\centering
\includegraphics[width=5.00968in,height=2.81677in]{media/image8.png}
\caption{\protect\phantomsection\label{_Toc144842668}{}\textbf{Figure
5.1.5: Read Section//Literature}}
\end{figure}

\begin{figure}
\centering
\includegraphics[width=5.77071in,height=3.24466in]{media/image9.png}
\caption{\protect\phantomsection\label{_Toc144842669}{}\textbf{Figure
5.1.6:
Read\textbackslash\textbackslash Literature\textbackslash\textbackslash Poems}}
\end{figure}

\subsection{}\label{section}

\subsection{5.2 WORK REMAINING}\label{work-remaining}

We have completed the basic structure of the application and
accomplished the CLI codes in several other modules. Some major tasks
that remains includes:

\begin{itemize}
\item
  Sign up module
\item
  Notes Section
\item
  Collaborator Section
\item
  User Management Section
\end{itemize}

\section{CHAPTER 6}\label{chapter-6}

\section{REFERENCES}\label{references}

~

{[}1{]}. en.wikipedia.org/wiki/Evernote~

{[}2{]}.\href{https://support.google.com/keep/answer/2888240?hl=en&co=GENIE.Platform\%3DAndroid}{support.google.com/keep/answer/2888240?hl=en\&co=GENIE.Platform\%3DAndroid}~

{[}3{]} https://github.com/kostasthanos/Social-Network-in-Cpp
