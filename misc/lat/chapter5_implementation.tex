\chapter{IMPLEMENTATION}

\section{Development Environment Setup}

\subsection{Tools and Technologies}

The development environment consists of:
\begin{itemize}
    \item \textbf{Operating System:} Windows 11
    \item \textbf{IDE:} Visual Studio Code with Python extension
    \item \textbf{Python:} Version 3.12
    \item \textbf{Django:} Version 5.2.8
    \item \textbf{Database:} PostgreSQL 13
    \item \textbf{Version Control:} Git and GitHub
\end{itemize}

\subsection{Project Structure}

The project follows Django's standard structure:

\begin{lstlisting}
BDIMS/
|-- accounts/           # Authentication app
|-- admin_panel/        # Admin features
|-- donor/              # Donor features
|-- blood_donation/     # Main project settings
|-- static/             # CSS, JS, images
|-- templates/          # HTML templates
|-- utils/              # Helper functions
|-- manage.py           # Django management script
`-- requirements.txt    # Dependencies
\end{lstlisting}

\section{Core Module Implementation}

\subsection{Models Implementation}

\subsubsection{Donor Model}

\begin{lstlisting}[language=Python]
from django.db import models
from django.contrib.auth.models import User

class Donor(models.Model):
    BLOOD_GROUPS = [
        ('A+', 'A+'), ('A-', 'A-'),
        ('B+', 'B+'), ('B-', 'B-'),
        ('O+', 'O+'), ('O-', 'O-'),
        ('AB+', 'AB+'), ('AB-', 'AB-'),
    ]
    
    user = models.OneToOneField(User, on_delete=models.CASCADE)
    blood_group = models.CharField(max_length=3, choices=BLOOD_GROUPS)
    date_of_birth = models.DateField()
    gender = models.CharField(max_length=10)
    phone_number = models.CharField(max_length=15)
    address = models.TextField()
    city = models.CharField(max_length=100)
    latitude = models.DecimalField(max_digits=10, decimal_places=8, null=True)
    longitude = models.DecimalField(max_digits=11, decimal_places=8, null=True)
    last_donation_date = models.DateField(null=True, blank=True)
    created_at = models.DateTimeField(auto_now_add=True)
    updated_at = models.DateTimeField(auto_now=True)
    
    def __str__(self):
        return f"{self.user.get_full_name()} - {self.blood_group}"
    
    def is_eligible(self):
        """Check if donor is eligible to donate"""
        if not self.last_donation_date:
            return True
        days_since = (timezone.now().date() - self.last_donation_date).days
        return days_since >= 56
\end{lstlisting}

\subsubsection{HealthMetrics Model}

\begin{lstlisting}[language=Python]
class HealthMetrics(models.Model):
    donor = models.ForeignKey(Donor, on_delete=models.CASCADE, 
                             related_name='health_metrics')
    weight = models.DecimalField(max_digits=5, decimal_places=2)
    blood_pressure_systolic = models.IntegerField()
    blood_pressure_diastolic = models.IntegerField()
    hemoglobin_level = models.DecimalField(max_digits=4, decimal_places=2)
    recorded_at = models.DateTimeField(auto_now_add=True)
    
    class Meta:
        ordering = ['-recorded_at']
        verbose_name_plural = "Health Metrics"
    
    def __str__(self):
        return f"{self.donor.user.username} - {self.recorded_at}"
\end{lstlisting}

\subsubsection{Hospital Model}

\begin{lstlisting}[language=Python]
class Hospital(models.Model):
    HOSPITAL_TYPES = [
        ('government', 'Government Hospital'),
        ('private', 'Private Hospital'),
        ('blood_bank', 'Blood Bank'),
    ]
    
    admin_user = models.OneToOneField(User, on_delete=models.CASCADE)
    name = models.CharField(max_length=200)
    phone = models.CharField(max_length=15)
    email = models.EmailField()
    address = models.TextField()
    city = models.CharField(max_length=100)
    hospital_type = models.CharField(max_length=20, choices=HOSPITAL_TYPES)
    operating_hours = models.CharField(max_length=100, default="24/7")
    created_at = models.DateTimeField(auto_now_add=True)
    
    def __str__(self):
        return self.name
\end{lstlisting}

\subsection{Views Implementation}

\subsubsection{Registration View}

\begin{lstlisting}[language=Python]
from django.contrib.auth import login
from django.contrib.auth.models import User
from django.shortcuts import render, redirect

@ratelimit(key='ip', rate='5/h', method='POST')
def register_view(request):
    if request.method == 'POST':
        # Get form data
        username = request.POST.get('username')
        password = request.POST.get('password')
        role = request.POST.get('role')
        
        # Create user
        user = User.objects.create_user(
            username=username,
            password=password,
            first_name=request.POST.get('first_name'),
            last_name=request.POST.get('last_name'),
            email=request.POST.get('email')
        )
        
        if role == 'donor':
            # Create donor profile
            Donor.objects.create(
                user=user,
                blood_group=request.POST.get('blood_group'),
                date_of_birth=request.POST.get('dob'),
                gender=request.POST.get('gender'),
                phone_number=request.POST.get('phone'),
                address=request.POST.get('address'),
                city=request.POST.get('city')
            )
            login(request, user)
            return redirect('donor:dashboard')
        
        elif role == 'admin':
            # Set staff status
            user.is_staff = True
            user.save()
            
            # Create hospital
            hospital = Hospital.objects.create(
                admin_user=user,
                name=request.POST.get('hospital_name'),
                phone=request.POST.get('hospital_phone'),
                email=request.POST.get('hospital_email'),
                address=request.POST.get('hospital_address'),
                city=request.POST.get('hospital_city'),
                hospital_type=request.POST.get('hospital_type')
            )
            
            # Initialize blood inventory
            blood_groups = ['A+', 'A-', 'B+', 'B-', 'O+', 'O-', 'AB+', 'AB-']
            for bg in blood_groups:
                BloodInventory.objects.create(
                    hospital=hospital,
                    blood_group=bg,
                    units_available=0
                )
            
            login(request, user)
            return redirect('admin_panel:dashboard')
    
    return render(request, 'accounts/register.html')
\end{lstlisting}

\subsubsection{Donor Dashboard View}

\begin{lstlisting}[language=Python]
from django.contrib.auth.decorators import login_required
from django.db.models import Count

@login_required
def donor_dashboard(request):
    donor = request.user.donor
    
    # Get statistics
    total_donations = DonationHistory.objects.filter(donor=donor).count()
    pending_requests = DonationRequest.objects.filter(
        donor=donor, 
        status='pending'
    ).count()
    
    # Get recent donations
    recent_donations = DonationHistory.objects.filter(
        donor=donor
    ).order_by('-donation_date')[:5]
    
    # Check eligibility
    is_eligible = donor.is_eligible()
    
    # Get latest health metrics
    latest_metrics = donor.health_metrics.first()
    
    context = {
        'donor': donor,
        'total_donations': total_donations,
        'pending_requests': pending_requests,
        'recent_donations': recent_donations,
        'is_eligible': is_eligible,
        'latest_metrics': latest_metrics,
    }
    
    return render(request, 'donor/dashboard.html', context)
\end{lstlisting}

\subsubsection{Schedule Donation View}

\begin{lstlisting}[language=Python]
@login_required
def schedule_donation(request):
    donor = request.user.donor
    
    if request.method == 'POST':
        # Check eligibility
        if not donor.is_eligible():
            messages.error(request, 'You are not eligible to donate yet.')
            return redirect('donor:dashboard')
        
        # Create donation request
        DonationRequest.objects.create(
            donor=donor,
            hospital_id=request.POST.get('hospital'),
            requested_date=request.POST.get('donation_date'),
            notes=request.POST.get('notes'),
            status='pending'
        )
        
        messages.success(request, 'Donation request submitted successfully!')
        return redirect('donor:donation_history')
    
    # Get all hospitals
    hospitals = Hospital.objects.all()
    
    context = {
        'hospitals': hospitals,
        'is_eligible': donor.is_eligible(),
    }
    
    return render(request, 'donor/schedule_donation.html', context)
\end{lstlisting}

\subsection{Template Implementation}

\subsubsection{Base Template}

\begin{lstlisting}[language=HTML]
<!DOCTYPE html>
<html lang="en">
<head>
    <meta charset="UTF-8">
    <meta name="viewport" content="width=device-width, initial-scale=1.0">
    <title>BDIMS</title>
    <link rel="stylesheet" href="">
    
</head>
<body>
    <nav class="navbar">
        <div class="container">
            <a href="/" class="logo">BDIMS</a>
            <ul class="nav-links">
                
                    <li><a href="">Logout</a></li>
                
                    <li><a href="">Login</a></li>
                    <li><a href="">Register</a></li>
                
            </ul>
        </div>
    </nav>
    
    <main class="content">
        
            
                <div class="alert alert-{{ message.tags }}">
                    {{ message }}
                </div>
            
        
        
        
    </main>
    
    <footer>
        <p>&copy; 2025 BDIMS. All rights reserved.</p>
    </footer>
    
    
</body>
</html>
\end{lstlisting}

\subsubsection{Donor Dashboard Template}

\begin{lstlisting}[language=HTML]



<div class="dashboard">
    <h1>Welcome, {{ user.get_full_name }}</h1>
    
    <div class="stats-grid">
        <div class="stat-card">
            <h3>Total Donations</h3>
            <p class="stat-number">{{ total_donations }}</p>
        </div>
        
        <div class="stat-card">
            <h3>Pending Requests</h3>
            <p class="stat-number">{{ pending_requests }}</p>
        </div>
        
        <div class="stat-card">
            <h3>Eligibility Status</h3>
            
                <p class="status eligible">Eligible</p>
            
                <p class="status not-eligible">Not Eligible</p>
            
        </div>
    </div>
    
    <div class="actions">
        <a href="" class="btn btn-primary">
            Schedule Donation
        </a>
        <a href="" class="btn btn-secondary">
            Update Profile
        </a>
    </div>
    
    <div class="recent-donations">
        <h2>Recent Donations</h2>
        <table>
            <thead>
                <tr>
                    <th>Date</th>
                    <th>Hospital</th>
                    <th>Units</th>
                </tr>
            </thead>
            <tbody>
                
                <tr>
                    <td>{{ donation.donation_date }}</td>
                    <td>{{ donation.hospital_name }}</td>
                    <td>{{ donation.units_donated }}</td>
                </tr>
                
                <tr>
                    <td colspan="3">No donations yet</td>
                </tr>
                
            </tbody>
        </table>
    </div>
</div>

\end{lstlisting}

\section{Interactive Map Implementation}

\subsection{Leaflet.js Integration}

\begin{lstlisting}[language=JavaScript]
// Initialize map
const map = L.map('map').setView([27.7172, 85.3240], 13);

// Add tile layer
L.tileLayer('https://{s}.tile.openstreetmap.org/{z}/{x}/{y}.png', {
    attribution: '&copy; OpenStreetMap contributors'
}).addTo(map);

// Marker variable
let marker;

// Click to place marker
map.on('click', function(e) {
    const lat = e.latlng.lat;
    const lng = e.latlng.lng;
    
    // Remove existing marker
    if (marker) {
        map.removeLayer(marker);
    }
    
    // Add new marker
    marker = L.marker([lat, lng]).addTo(map);
    
    // Update hidden form fields
    document.getElementById('latitude').value = lat;
    document.getElementById('longitude').value = lng;
    
    // Reverse geocode
    reverseGeocode(lat, lng);
});

// Reverse geocoding function
function reverseGeocode(lat, lng) {
    fetch(`https://nominatim.openstreetmap.org/reverse?lat=${lat}&lon=${lng}&format=json`)
        .then(response => response.json())
        .then(data => {
            document.getElementById('address').value = data.display_name;
        });
}

// Get current location
document.getElementById('getCurrentLocation').addEventListener('click', function() {
    if (navigator.geolocation) {
        navigator.geolocation.getCurrentPosition(function(position) {
            const lat = position.coords.latitude;
            const lng = position.coords.longitude;
            
            map.setView([lat, lng], 15);
            
            if (marker) {
                map.removeLayer(marker);
            }
            
            marker = L.marker([lat, lng]).addTo(map);
            
            document.getElementById('latitude').value = lat;
            document.getElementById('longitude').value = lng;
            
            reverseGeocode(lat, lng);
        });
    }
});
\end{lstlisting}

\section{Database Configuration}

\subsection{PostgreSQL Setup}

\begin{lstlisting}[language=Python]
# settings.py
DATABASES = {
    'default': {
        'ENGINE': 'django.db.backends.postgresql',
        'NAME': '1',
        'USER': 'postgres',
        'PASSWORD': '0',
        'HOST': 'localhost',
        'PORT': '5432',
    }
}
\end{lstlisting}

\subsection{Migrations}

\begin{lstlisting}[language=bash]
# Create migrations
python manage.py makemigrations

# Apply migrations
python manage.py migrate

# Create superuser
python manage.py createsuperuser
\end{lstlisting}

\section{Security Implementation}

\subsection{CSRF Protection}

All forms include CSRF tokens:
\begin{lstlisting}[language=HTML]
<form method="post">
    
    <!-- form fields -->
</form>
\end{lstlisting}

\subsection{Password Hashing}

Django automatically hashes passwords using PBKDF2:
\begin{lstlisting}[language=Python]
# In settings.py
PASSWORD_HASHERS = [
    'django.contrib.auth.hashers.PBKDF2PasswordHasher',
    'django.contrib.auth.hashers.PBKDF2SHA1PasswordHasher',
]
\end{lstlisting}

\subsection{Rate Limiting}

Implemented using django-ratelimit:
\begin{lstlisting}[language=Python]
from django_ratelimit.decorators import ratelimit

@ratelimit(key='ip', rate='5/h', method='POST')
def register_view(request):
    # Registration logic
    pass
\end{lstlisting}

\section{Deployment Considerations}

\subsection{Production Settings}

\begin{lstlisting}[language=Python]
# Production settings
DEBUG = False
ALLOWED_HOSTS = ['yourdomain.com', 'www.yourdomain.com']

# Static files
STATIC_ROOT = os.path.join(BASE_DIR, 'staticfiles')

# Security
SECURE_SSL_REDIRECT = True
SESSION_COOKIE_SECURE = True
CSRF_COOKIE_SECURE = True
\end{lstlisting}

\subsection{Server Configuration}

For production deployment:
\begin{itemize}
    \item Use Gunicorn or uWSGI as WSGI server
    \item Configure Nginx as reverse proxy
    \item Set up SSL certificates
    \item Configure automated backups
    \item Implement monitoring and logging
\end{itemize}
