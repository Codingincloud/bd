\documentclass[12pt,a4paper]{report}
\usepackage[utf8]{inputenc}
\usepackage[top=1in,bottom=1in,left=1.5in,right=1in]{geometry}
\usepackage{newtxtext,newtxmath}
\usepackage[hidelinks]{hyperref}
\usepackage{amsmath}
\usepackage{graphicx}
\usepackage{booktabs}
\usepackage{xcolor}

\begin{document}

\begin{titlepage}
    \centering
    \vspace*{2cm}
    {\Large \textbf{KHWOPA ENGINEERING COLLEGE}}\\[0.5cm]
    {\large Libali-08, Bhaktapur, Nepal}\\[0.3cm]
    {\large (Affiliated to Purbanchal University)}\\[2cm]
    
    {\LARGE \textbf{BLOOD DONOR INFORMATION}}\\[0.2cm]
    {\LARGE \textbf{MANAGEMENT SYSTEM}}\\[0.5cm]
    
    {\large \textit{A Final Defense Report}}\\[2cm]
    
    {\large \textbf{Submitted By:}}\\[0.5cm]
    \begin{tabular}{ll}
        Bishal Shrestha & (790310)\\
        Chirayu Shrestha & (790311)\\
        Pappu Yadav & (790324)\\
        Prashant Ghimire & (790328)
    \end{tabular}\\[2cm]
    
    {\large \textbf{Supervised By:}}\\[0.3cm]
    {\large Er. Anish Baral}\\[0.2cm]
    {\large Department of Computer Engineering}\\[2cm]
    
    \vfill
    
    {\large \textbf{November 2025}}
    
\end{titlepage}

\pagenumbering{roman}
\setcounter{page}{1}

\chapter*{ABSTRACT}
\addcontentsline{toc}{chapter}{ABSTRACT}

The Blood Donor Information Management System (BDIMS) is a comprehensive web-based application designed to modernize and streamline blood donation management for healthcare institutions in Nepal. The system addresses critical challenges in donor information management, blood inventory tracking, and emergency response coordination through a centralized digital platform.

Built using Django framework and PostgreSQL database, BDIMS provides role-based access for both administrators and donors. The system enables administrators to manage comprehensive donor databases, track real-time blood inventory, handle emergency blood requests, and generate detailed analytical reports.

\textbf{Keywords:} Blood Donation Management, Django Framework, PostgreSQL Database, Emergency Response System, Healthcare Information System

\tableofcontents
\listoffigures
\listoftables

\cleardoublepage
\pagenumbering{arabic}
\setcounter{page}{1}

\chapter{INTRODUCTION}

\section{Background}

In the modern healthcare landscape, efficient management of blood donation systems is crucial for saving lives and ensuring the availability of safe blood supplies. Healthcare institutions, particularly in developing countries like Nepal, face significant challenges in managing comprehensive donor information including medical records, health metrics, donation history, and eligibility tracking.

Nepal, with a population of approximately 30 million people, requires an estimated 300,000 units of blood annually according to the Ministry of Health and Population. However, the gap between supply and demand remains significant, particularly in rural areas and during emergency situations.

The Blood Donor Information Management System (BDIMS) has been developed as a comprehensive solution to address these critical challenges. Built using modern web technologies including the Django framework and PostgreSQL database, BDIMS provides a centralized platform for efficiently organizing, storing, and managing all donor-related information.

\section{Problem Statement}

Healthcare institutions face multifaceted challenges in managing comprehensive donor information:

\begin{itemize}
    \item Scattered medical records and donation histories
    \item Manual health metrics tracking prone to errors
    \item Inefficient donor search and matching systems
    \item Lack of centralized databases across institutions
    \item Complex eligibility calculations requiring multiple criteria
    \item Delays in emergency response coordination
    \item Inadequate inventory management leading to shortages or wastage
\end{itemize}

\section{Objectives}

The primary objective is to develop a comprehensive Blood Donor Information Management System that:

\begin{itemize}
    \item Centralizes donor information with detailed profiles
    \item Automates eligibility calculations based on medical guidelines
    \item Provides real-time blood inventory tracking
    \item Enables rapid emergency blood request matching
    \item Integrates interactive maps for location-based services
    \item Generates comprehensive analytics and reports
    \item Ensures data security through role-based access control
\end{itemize}

\chapter{LITERATURE REVIEW}

Blood Donor Information Management Systems have emerged as critical components of modern healthcare infrastructure. Research by Kumar and Gupta (2020) shows that digital systems reduced administrative workload by over 30\% and improved emergency response times by approximately 45\%.

In Nepal, the Nepal Red Cross Society (NRCS) has been progressively modernizing operations. The Hamro LifeBank project, launched in 2022, represents a significant milestone, reducing emergency response times from 3.5 hours to 45 minutes in the Kathmandu Valley.

International best practices identified by Ali et al. (2022) include:
\begin{itemize}
    \item User-centric design showing 20-25\% higher donor retention
    \item Mobile accessibility with 40\% higher engagement rates
    \item Automated notifications increasing response rates by 55\%
    \item Gamification increasing repeat donations by 32\%
\end{itemize}

\chapter{PROJECT MANAGEMENT}

\section{Team Members}

The project was carried out by four Computer Engineering students:

\begin{itemize}
    \item Bishal Shrestha (790310) - Backend Development, Database Design
    \item Chirayu Shrestha (790311) - Frontend Development, UI/UX Design
    \item Pappu Yadav (790324) - System Integration, Testing
    \item Prashant Ghimire (790328) - Documentation, API Development
\end{itemize}

Supervised by Er. Anish Baral, Department of Computer Engineering, Khwopa Engineering College.

\section{Feasibility Study}

\subsection{Economic Feasibility}

The project demonstrates excellent economic viability:
\begin{itemize}
    \item Zero software licensing costs (open-source technologies)
    \item Minimal hardware requirements
    \item Low hosting costs (\$20-50 monthly)
    \item Estimated ROI: 200-450\%
\end{itemize}

\subsection{Technical Feasibility}

Utilizes proven, stable technologies:
\begin{itemize}
    \item Django Framework (15+ years mature)
    \item PostgreSQL Database (enterprise-grade)
    \item Modern web standards (HTML5, CSS3, JavaScript)
    \item Leaflet.js for mapping
\end{itemize}

\chapter{CONCLUSION}

The Blood Donor Information Management System successfully addresses critical challenges in blood donation management through a comprehensive, user-friendly digital platform. The system has been developed using modern web technologies and follows software engineering best practices.

Key achievements include:
\begin{itemize}
    \item Centralized donor information management
    \item Automated eligibility tracking
    \item Real-time inventory monitoring
    \item Emergency request coordination
    \item Interactive location-based services
    \item Comprehensive reporting and analytics
\end{itemize}

The system is ready for deployment and has the potential to significantly improve blood donation coordination in Nepal, ultimately contributing to saving lives through efficient blood supply management.

\chapter*{REFERENCES}
\addcontentsline{toc}{chapter}{REFERENCES}

\begin{enumerate}
    \item Kumar, A. \& Gupta, S. (2020). Blood Donation Management Systems: A Comprehensive Review. Healthcare Informatics Research, 26(3), 145-158.
    \item Ali, H., Ahmed, F., \& Khan, Z. (2022). Usability-Focused Systems for Blood Donor Retention. International Journal of Medical Informatics, 158, 104651.
    \item Sharma, P., Patel, R., \& Singh, M. (2019). Digital Donor Registries in South Asia. Journal of Healthcare Management, 15(2), 78-92.
    \item Ministry of Health and Population (2022). Blood Transfusion Services in Nepal: Annual Report 2022. Government of Nepal.
\end{enumerate}

\end{document}
