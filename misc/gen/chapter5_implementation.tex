\chapter{IMPLEMENTATION}

\section{Introduction}

This chapter describes the implementation of the Blood Donor Information Management System (BDIMS), including the development environment setup, technology stack, module-by-module implementation details, code structure, and key algorithms.

\section{Development Environment}

\subsection{Hardware Configuration}

\textbf{Development Machines:}
\begin{itemize}
    \item Processor: Intel Core i5-8th Gen / AMD Ryzen 5
    \item RAM: 8-16 GB DDR4
    \item Storage: 256 GB SSD
    \item Display: 1920x1080 resolution
    \item Internet: 10 Mbps broadband connection
\end{itemize}

\subsection{Software Tools}

\begin{table}[h]
\centering
\caption{Development Tools and Versions}
\begin{tabular}{|p{4cm}|p{3cm}|p{5cm}|}
\hline
\textbf{Tool} & \textbf{Version} & \textbf{Purpose} \\
\hline
Operating System & Windows 10/11 & Development platform \\
\hline
Python & 3.12.0 & Programming language \\
\hline
Django & 5.2.8 & Web framework \\
\hline
PostgreSQL & 15.3 & Database system \\
\hline
VS Code & 1.85.0 & Code editor/IDE \\
\hline
Git & 2.42.0 & Version control \\
\hline
GitHub & N/A & Remote repository \\
\hline
Chrome DevTools & Latest & Frontend debugging \\
\hline
Postman & 10.18 & API testing \\
\hline
pgAdmin 4 & 7.8 & Database management \\
\hline
\end{tabular}
\end{table}

\subsection{VS Code Extensions}

\begin{itemize}
    \item Python (Microsoft)
    \item Pylance
    \item Django Template
    \item GitLens
    \item HTML CSS Support
    \item JavaScript (ES6) code snippets
    \item Prettier - Code formatter
\end{itemize}

\section{Technology Stack}

\subsection{Backend Technologies}

\subsubsection{Django Framework 5.2.8}

Django is a high-level Python web framework that enables rapid development of secure and maintainable websites.

\textbf{Key Features Used:}
\begin{itemize}
    \item \textbf{ORM (Object-Relational Mapping):} Database operations without writing SQL
    \item \textbf{Admin Interface:} Built-in admin panel for data management
    \item \textbf{Authentication:} User registration, login, session management
    \item \textbf{Forms:} Form handling with validation
    \item \textbf{Templates:} Dynamic HTML rendering with template inheritance
    \item \textbf{URL Routing:} Clean URL patterns
    \item \textbf{Security:} CSRF protection, password hashing, SQL injection prevention
\end{itemize}

\textbf{Django Architecture (MVT):}

\begin{lstlisting}[language=Python, caption=Django MVT Pattern Example]
# models.py (Model - Data Layer)
class Donor(models.Model):
    user = models.OneToOneField(User, on_delete=models.CASCADE)
    blood_group = models.CharField(max_length=3)
    # ... other fields

# views.py (View - Logic Layer)
def donor_dashboard(request):
    donor = request.user.donor
    donations = donor.donationhistory_set.all()[:5]
    context = {'donor': donor, 'donations': donations}
    return render(request, 'donor/dashboard.html', context)

# urls.py (URL Routing)
urlpatterns = [
    path('dashboard/', views.donor_dashboard, name='dashboard'),
]
\end{lstlisting}

\subsubsection{PostgreSQL 15.3}

PostgreSQL is an advanced open-source relational database management system.

\textbf{Why PostgreSQL:}
\begin{itemize}
    \item ACID compliance for data integrity
    \item Advanced data types (JSON, arrays)
    \item Robust indexing and query optimization
    \item Excellent performance with large datasets
    \item Strong community support
    \item Free and open-source
\end{itemize}

\textbf{Database Configuration:}

\begin{lstlisting}[language=Python, caption=Database Settings (settings.py)]
DATABASES = {
    'default': {
        'ENGINE': 'django.db.backends.postgresql',
        'NAME': 'bdims_db',
        'USER': 'postgres',
        'PASSWORD': 'your_password',
        'HOST': 'localhost',
        'PORT': '5432',
    }
}
\end{lstlisting}

\subsubsection{Python Libraries}

\begin{table}[h]
\centering
\caption{Python Dependencies}
\begin{tabular}{|p{3cm}|p{2cm}|p{7cm}|}
\hline
\textbf{Library} & \textbf{Version} & \textbf{Purpose} \\
\hline
psycopg2 & 2.9.9 & PostgreSQL database adapter \\
\hline
Pillow & 10.1.0 & Image processing for profile pictures \\
\hline
python-decouple & 3.8 & Environment variable management \\
\hline
whitenoise & 6.6.0 & Static file serving in production \\
\hline
\end{tabular}
\end{table}

\subsection{Frontend Technologies}

\subsubsection{HTML5}

Semantic HTML5 elements for structure:
\begin{itemize}
    \item \texttt{<header>}, \texttt{<nav>}, \texttt{<main>}, \texttt{<section>}, \texttt{<article>}, \texttt{<footer>}
    \item Form elements with validation attributes
    \item Accessible markup (ARIA labels where needed)
\end{itemize}

\subsubsection{CSS3}

Modern CSS features for styling:
\begin{itemize}
    \item CSS Grid and Flexbox for responsive layouts
    \item CSS Variables for theme consistency
    \item Media queries for mobile responsiveness
    \item Animations and transitions for smooth UX
    \item Custom properties for maintainability
\end{itemize}

\begin{lstlisting}[language=HTML, caption=CSS Variables for Theme]
:root {
    --color-primary-600: #DC2626;
    --color-primary-700: #B91C1C;
    --color-success-600: #16A34A;
    --space-4: 1rem;
    --radius-lg: 0.75rem;
    --shadow-md: 0 4px 6px rgba(0,0,0,0.1);
}
\end{lstlisting}

\subsubsection{JavaScript (ES6+)}

Client-side scripting for interactivity:
\begin{itemize}
    \item Form validation and real-time feedback
    \item AJAX requests for dynamic content
    \item DOM manipulation
    \item Event handling
\end{itemize}

\subsubsection{Leaflet.js 1.9.4}

Open-source JavaScript library for interactive maps:

\begin{lstlisting}[language=HTML, caption=Leaflet Map Initialization]
// Initialize map
const map = L.map('map').setView([27.6915, 85.3240], 13);

// Add OpenStreetMap tiles
L.tileLayer('https://{s}.tile.openstreetmap.org/{z}/{x}/{y}.png', {
    attribution: '&copy; OpenStreetMap contributors'
}).addTo(map);

// Add marker on click
map.on('click', function(e) {
    const { lat, lng } = e.latlng;
    updateDonorLocation(lat, lng);
});
\end{lstlisting}

\subsubsection{Chart.js 4.4.0}

JavaScript charting library for data visualization:

\begin{lstlisting}[language=HTML, caption=Blood Inventory Chart]
const ctx = document.getElementById('inventoryChart').getContext('2d');
new Chart(ctx, {
    type: 'bar',
    data: {
        labels: ['A+', 'A-', 'B+', 'B-', 'O+', 'O-', 'AB+', 'AB-'],
        datasets: [{
            label: 'Units Available',
            data: [120, 45, 98, 32, 156, 28, 67, 19],
            backgroundColor: 'rgba(220, 38, 38, 0.8)'
        }]
    },
    options: { responsive: true, maintainAspectRatio: false }
});
\end{lstlisting}

\subsubsection{Font Awesome 6.4.2}

Icon library for visual elements:
\begin{itemize}
    \item \texttt{fa-tint} for blood-related features
    \item \texttt{fa-heartbeat} for health metrics
    \item \texttt{fa-calendar} for scheduling
    \item \texttt{fa-user} for profile management
\end{itemize}

\section{Project Structure}

\subsection{Directory Organization}

\begin{lstlisting}[caption=BDIMS Project Structure]
BDIMS/
|-- accounts/              # Authentication module
|   |-- views.py          # Login, register, logout
|   |-- urls.py           # Auth URL patterns
|   |-- templates/        # Login/register templates
|
|-- admin_panel/          # Admin dashboard module
|   |-- views.py          # Admin functions
|   |-- urls.py           # Admin URL patterns
|   |-- templates/        # Admin templates
|
|-- donor/                # Donor functionality module
|   |-- models.py         # Data models
|   |-- views.py          # Donor views
|   |-- forms.py          # Form classes
|   |-- urls.py           # Donor URL patterns
|   |-- admin.py          # Admin configuration
|   |-- signals.py        # Django signals
|   |-- middleware.py     # Custom middleware
|   |-- templates/        # Donor templates
|   |-- migrations/       # Database migrations
|
|-- blood_donation/       # Project settings
|   |-- settings.py       # Django configuration
|   |-- urls.py           # Root URL patterns
|   |-- wsgi.py           # WSGI application
|   |-- asgi.py           # ASGI application
|
|-- utils/                # Utility functions
|   |-- constants.py      # System constants
|   |-- validators.py     # Custom validators
|   |-- decorators.py     # Custom decorators
|   |-- geocoding.py      # Location services
|   |-- notification_service.py  # Notifications
|
|-- static/               # Static files
|   |-- css/             # Stylesheets
|   |-- js/              # JavaScript files
|   |-- images/          # Image assets
|
|-- templates/            # Global templates
|   |-- home.html        # Landing page
|   |-- components/      # Reusable components
|
|-- media/               # User-uploaded files
|-- db.sqlite3           # Database file (dev)
|-- manage.py            # Django management script
|-- requirements.txt     # Python dependencies
\end{lstlisting}

\section{Module Implementation}

\subsection{Authentication Module (accounts/)}

\subsubsection{User Registration}

\begin{lstlisting}[language=Python, caption=Donor Registration View]
def register(request):
    if request.method == 'POST':
        # Extract form data
        username = request.POST.get('username')
        email = request.POST.get('email')
        password = request.POST.get('password')
        # ... other fields
        
        # Validate data
        if User.objects.filter(username=username).exists():
            messages.error(request, 'Username already exists')
            return redirect('accounts:register')
        
        # Create user
        user = User.objects.create_user(
            username=username,
            email=email,
            password=password,
            first_name=first_name,
            last_name=last_name
        )
        
        # Create donor profile
        Donor.objects.create(
            user=user,
            blood_group=blood_group,
            date_of_birth=date_of_birth,
            # ... other fields
        )
        
        messages.success(request, 'Registration successful!')
        return redirect('accounts:login')
    
    return render(request, 'accounts/register.html')
\end{lstlisting}

\subsubsection{User Login}

\begin{lstlisting}[language=Python, caption=Login View with Authentication]
from django.contrib.auth import authenticate, login

def user_login(request):
    if request.method == 'POST':
        username = request.POST.get('username')
        password = request.POST.get('password')
        
        # Authenticate user
        user = authenticate(request, username=username, password=password)
        
        if user is not None:
            login(request, user)
            
            # Redirect based on role
            if user.is_staff:
                return redirect('admin_panel:dashboard')
            else:
                return redirect('donor:dashboard')
        else:
            messages.error(request, 'Invalid credentials')
    
    return render(request, 'accounts/login.html')
\end{lstlisting}

\subsection{Donor Module (donor/)}

\subsubsection{Donor Model}

\begin{lstlisting}[language=Python, caption=Donor Model with Properties]
class Donor(models.Model):
    BLOOD_GROUPS = [
        ('A+', 'A+'), ('A-', 'A-'),
        ('B+', 'B+'), ('B-', 'B-'),
        ('O+', 'O+'), ('O-', 'O-'),
        ('AB+', 'AB+'), ('AB-', 'AB-'),
    ]
    
    user = models.OneToOneField(User, on_delete=models.CASCADE)
    blood_group = models.CharField(max_length=3, choices=BLOOD_GROUPS)
    date_of_birth = models.DateField()
    latitude = models.DecimalField(max_digits=10, decimal_places=8, 
                                  null=True, blank=True)
    longitude = models.DecimalField(max_digits=11, decimal_places=8, 
                                   null=True, blank=True)
    last_donation_date = models.DateField(null=True, blank=True)
    is_eligible = models.BooleanField(default=True)
    
    @property
    def age(self):
        today = date.today()
        return today.year - self.date_of_birth.year - (
            (today.month, today.day) < 
            (self.date_of_birth.month, self.date_of_birth.day)
        )
    
    @property
    def next_eligible_date(self):
        if self.last_donation_date:
            return self.last_donation_date + timedelta(days=56)
        return None
    
    def check_eligibility(self):
        """Check if donor is eligible to donate"""
        # Age check
        if not (18 <= self.age <= 65):
            return False, "Age must be between 18 and 65"
        
        # Weight check
        if self.weight < 50:
            return False, "Weight must be at least 50 kg"
        
        # Donation interval check (56 days)
        if self.last_donation_date:
            days_since = (date.today() - self.last_donation_date).days
            if days_since < 56:
                return False, f"Must wait {56 - days_since} more days"
        
        # Check recent health metrics
        latest_metrics = self.healthmetrics_set.first()
        if latest_metrics:
            if latest_metrics.hemoglobin_level < 12.5:
                return False, "Hemoglobin level too low"
            if latest_metrics.blood_pressure_systolic > 140:
                return False, "Blood pressure too high"
        
        return True, "Eligible to donate"
\end{lstlisting}

\subsubsection{Donor Dashboard}

\begin{lstlisting}[language=Python, caption=Donor Dashboard View]
@login_required
def donor_dashboard(request):
    try:
        donor = request.user.donor
    except Donor.DoesNotExist:
        messages.error(request, 'Donor profile not found')
        return redirect('accounts:login')
    
    # Get donation statistics
    total_donations = donor.donationhistory_set.count()
    last_donation = donor.donationhistory_set.first()
    
    # Get recent health metrics
    recent_metrics = donor.healthmetrics_set.all()[:5]
    
    # Check eligibility
    is_eligible, eligibility_message = donor.check_eligibility()
    
    # Get emergency requests matching blood type
    emergency_requests = EmergencyRequest.objects.filter(
        blood_group=donor.blood_group,
        is_active=True
    )
    
    # Get pending donation requests
    pending_requests = DonationRequest.objects.filter(
        donor=donor,
        status='pending'
    )
    
    context = {
        'donor': donor,
        'total_donations': total_donations,
        'last_donation': last_donation,
        'recent_metrics': recent_metrics,
        'is_eligible': is_eligible,
        'eligibility_message': eligibility_message,
        'emergency_requests': emergency_requests,
        'pending_requests': pending_requests,
    }
    
    return render(request, 'donor/dashboard.html', context)
\end{lstlisting}

\subsubsection{Health Metrics Tracking}

\begin{lstlisting}[language=Python, caption=Add Health Metrics View]
@login_required
def add_health_metrics(request):
    donor = request.user.donor
    
    if request.method == 'POST':
        hemoglobin = request.POST.get('hemoglobin_level')
        bp_systolic = request.POST.get('bp_systolic')
        bp_diastolic = request.POST.get('bp_diastolic')
        weight = request.POST.get('weight')
        temperature = request.POST.get('temperature')
        pulse_rate = request.POST.get('pulse_rate')
        
        # Create health metrics record
        HealthMetrics.objects.create(
            donor=donor,
            hemoglobin_level=hemoglobin,
            blood_pressure_systolic=bp_systolic,
            blood_pressure_diastolic=bp_diastolic,
            weight=weight,
            temperature=temperature,
            pulse_rate=pulse_rate
        )
        
        # Update donor weight
        donor.weight = weight
        donor.save()
        
        messages.success(request, 'Health metrics added successfully')
        return redirect('donor:dashboard')
    
    # Get previous metrics for reference
    previous_metrics = donor.healthmetrics_set.all()[:10]
    
    return render(request, 'donor/add_health_metrics.html', {
        'donor': donor,
        'previous_metrics': previous_metrics
    })
\end{lstlisting}

\subsection{Inventory Management}

\subsubsection{Blood Inventory Model}

\begin{lstlisting}[language=Python, caption=Blood Inventory with Signals]
class BloodInventory(models.Model):
    hospital = models.ForeignKey('Hospital', on_delete=models.CASCADE)
    blood_group = models.CharField(max_length=5, choices=Donor.BLOOD_GROUPS)
    units_available = models.FloatField(default=0.0)
    units_reserved = models.FloatField(default=0.0)
    last_updated = models.DateTimeField(auto_now=True)
    
    class Meta:
        unique_together = [['hospital', 'blood_group']]
    
    @property
    def status_level(self):
        """Return inventory status"""
        if self.units_available < 10:
            return 'critical'
        elif self.units_available < 25:
            return 'low'
        elif self.units_available < 50:
            return 'medium'
        else:
            return 'good'
    
    @property
    def total_units(self):
        return self.units_available + self.units_reserved

# Signal to auto-update inventory on donation
@receiver(post_save, sender=DonationHistory)
def update_inventory_on_donation(sender, instance, created, **kwargs):
    if created and instance.hospital:
        inventory, _ = BloodInventory.objects.get_or_create(
            hospital=instance.hospital,
            blood_group=instance.donor.blood_group
        )
        inventory.units_available += float(instance.units_donated)
        inventory.save()
\end{lstlisting}

\subsection{Emergency Response System}

\begin{lstlisting}[language=Python, caption=Emergency Request Matching]
@login_required
@user_passes_test(lambda u: u.is_staff)
def create_emergency_request(request):
    if request.method == 'POST':
        blood_group = request.POST.get('blood_group')
        units_needed = request.POST.get('units_needed')
        urgency_level = request.POST.get('urgency_level')
        hospital_id = request.POST.get('hospital')
        
        # Create emergency request
        emergency = EmergencyRequest.objects.create(
            blood_group=blood_group,
            units_needed=units_needed,
            urgency_level=urgency_level,
            hospital_id=hospital_id,
            created_by=request.user
        )
        
        # Find matching eligible donors
        matching_donors = Donor.objects.filter(
            blood_group=blood_group,
            is_eligible=True,
            allow_emergency_contact=True
        )
        
        # Filter by proximity (if hospital has coordinates)
        if emergency.hospital.latitude and emergency.hospital.longitude:
            nearby_donors = []
            for donor in matching_donors:
                if donor.latitude and donor.longitude:
                    distance = donor.distance_to(
                        emergency.hospital.latitude,
                        emergency.hospital.longitude
                    )
                    if distance and distance < 50:  # Within 50km
                        nearby_donors.append(donor)
            matching_donors = nearby_donors
        
        # Notify donors (placeholder for future SMS/email)
        donor_count = len(matching_donors)
        messages.success(
            request, 
            f'Emergency request created. {donor_count} matching donors found.'
        )
        
        return redirect('admin_panel:emergency_requests')
    
    hospitals = Hospital.objects.all()
    return render(request, 'admin_panel/create_emergency.html', {
        'hospitals': hospitals
    })
\end{lstlisting}

\subsection{Location Tracking Implementation}

\subsubsection{Interactive Map Update}

\begin{lstlisting}[language=HTML, caption=Leaflet Map for Location Update]
<div id="map" style="height: 500px;"></div>

<script>
// Initialize map centered on Kathmandu
const map = L.map('map').setView([27.6915, 85.3240], 13);

// Add tile layer
L.tileLayer('https://{s}.tile.openstreetmap.org/{z}/{x}/{y}.png').addTo(map);

let marker;

// GPS auto-detect
document.getElementById('gps-btn').onclick = function() {
    navigator.geolocation.getCurrentPosition(function(position) {
        const lat = position.coords.latitude;
        const lng = position.coords.longitude;
        updateLocation(lat, lng);
    });
};

// Click to select location
map.on('click', function(e) {
    updateLocation(e.latlng.lat, e.latlng.lng);
});

function updateLocation(lat, lng) {
    // Update or create marker
    if (marker) {
        marker.setLatLng([lat, lng]);
    } else {
        marker = L.marker([lat, lng]).addTo(map);
    }
    
    // Update form fields
    document.getElementById('id_latitude').value = lat;
    document.getElementById('id_longitude').value = lng;
    
    // Reverse geocode to get address
    fetch(`https://nominatim.openstreetmap.org/reverse?format=json&lat=${lat}&lon=${lng}`)
        .then(res => res.json())
        .then(data => {
            document.getElementById('address-display').innerText = data.display_name;
        });
    
    map.setView([lat, lng], 15);
}
</script>
\end{lstlisting}

\section{Key Algorithms}

\subsection{Eligibility Calculation Algorithm}

\begin{lstlisting}[language=Python, caption=Comprehensive Eligibility Check]
def calculate_eligibility(donor):
    """
    Calculate donor eligibility based on multiple criteria.
    Returns (is_eligible: bool, message: str, details: dict)
    """
    details = {}
    
    # 1. Age check (18-65 years)
    age = donor.age
    if not (18 <= age <= 65):
        return False, f"Age ({age}) must be 18-65", details
    details['age'] = {'status': 'pass', 'value': age}
    
    # 2. Weight check (>= 50 kg)
    if donor.weight < 50:
        return False, f"Weight ({donor.weight} kg) must be >= 50 kg", details
    details['weight'] = {'status': 'pass', 'value': float(donor.weight)}
    
    # 3. Donation interval check (56 days)
    if donor.last_donation_date:
        days_since = (date.today() - donor.last_donation_date).days
        if days_since < 56:
            next_date = donor.last_donation_date + timedelta(days=56)
            return False, f"Must wait {56 - days_since} days (until {next_date})", details
        details['interval'] = {'status': 'pass', 'days_since': days_since}
    else:
        details['interval'] = {'status': 'pass', 'days_since': None}
    
    # 4. Health metrics check
    latest_metrics = donor.healthmetrics_set.first()
    if latest_metrics:
        # Hemoglobin check (>= 12.5 g/dL for females, >= 13.0 for males)
        min_hb = 13.0 if donor.gender == 'M' else 12.5
        if latest_metrics.hemoglobin_level < min_hb:
            return False, f"Hemoglobin ({latest_metrics.hemoglobin_level} g/dL) too low", details
        details['hemoglobin'] = {'status': 'pass', 'value': float(latest_metrics.hemoglobin_level)}
        
        # Blood pressure check (90-140 systolic, 60-90 diastolic)
        if not (90 <= latest_metrics.blood_pressure_systolic <= 140):
            return False, "Blood pressure outside acceptable range", details
        if not (60 <= latest_metrics.blood_pressure_diastolic <= 90):
            return False, "Blood pressure outside acceptable range", details
        details['blood_pressure'] = {'status': 'pass', 'value': f"{latest_metrics.blood_pressure_systolic}/{latest_metrics.blood_pressure_diastolic}"}
    
    # All checks passed
    return True, "Eligible to donate blood", details
\end{lstlisting}

\subsection{Distance Calculation (Haversine Formula)}

\begin{lstlisting}[language=Python, caption=Geographic Distance Calculation]
import math

def haversine_distance(lat1, lon1, lat2, lon2):
    """
    Calculate the great-circle distance between two points
    on Earth using Haversine formula.
    Returns distance in kilometers.
    """
    R = 6371  # Earth's radius in kilometers
    
    # Convert degrees to radians
    lat1_rad = math.radians(lat1)
    lat2_rad = math.radians(lat2)
    delta_lat = math.radians(lat2 - lat1)
    delta_lon = math.radians(lon2 - lon1)
    
    # Haversine formula
    a = (math.sin(delta_lat / 2) ** 2 +
         math.cos(lat1_rad) * math.cos(lat2_rad) *
         math.sin(delta_lon / 2) ** 2)
    
    c = 2 * math.asin(math.sqrt(a))
    
    distance = R * c
    return round(distance, 2)

# Usage in Donor model
def distance_to(self, other_lat, other_lng):
    if not (self.latitude and self.longitude and other_lat and other_lng):
        return None
    return haversine_distance(
        float(self.latitude), float(self.longitude),
        float(other_lat), float(other_lng)
    )
\end{lstlisting}

\section{Security Implementation}

\subsection{CSRF Protection}

\begin{lstlisting}[language=HTML, caption=CSRF Token in Forms]
<form method="POST">
    
    <input type="text" name="username" required>
    <input type="password" name="password" required>
    <button type="submit">Login</button>
</form>
\end{lstlisting}

\subsection{Custom Decorators}

\begin{lstlisting}[language=Python, caption=Role-Based Access Decorator]
from functools import wraps
from django.shortcuts import redirect
from django.contrib import messages

def donor_required(view_func):
    """Decorator to require donor role"""
    @wraps(view_func)
    def wrapper(request, *args, **kwargs):
        if not request.user.is_authenticated:
            return redirect('accounts:login')
        
        if request.user.is_staff:
            messages.error(request, 'This page is for donors only')
            return redirect('admin_panel:dashboard')
        
        try:
            donor = request.user.donor
        except Donor.DoesNotExist:
            messages.error(request, 'Donor profile not found')
            return redirect('accounts:login')
        
        return view_func(request, *args, **kwargs)
    return wrapper

# Usage
@donor_required
def donor_dashboard(request):
    # View implementation
    pass
\end{lstlisting}

\section{Database Migrations}

\begin{lstlisting}[language=Python, caption=Creating and Applying Migrations]
# Generate migrations from model changes
python manage.py makemigrations

# Apply migrations to database
python manage.py migrate

# View migration SQL (without applying)
python manage.py sqlmigrate donor 0001

# List all migrations
python manage.py showmigrations
\end{lstlisting}

\section{Challenges and Solutions}

\subsection{Challenge 1: Real-time Eligibility Checking}

\textbf{Problem:} Calculating eligibility requires checking multiple conditions across different tables (Donor, HealthMetrics, DonationHistory).

\textbf{Solution:} Implemented a comprehensive \texttt{check\_eligibility()} method in the Donor model that aggregates all checks in a single database-efficient query, with caching of results in the session.

\subsection{Challenge 2: Location-based Donor Matching}

\textbf{Problem:} Finding nearby donors requires calculating distances between geographic coordinates for potentially thousands of donors.

\textbf{Solution:} Implemented Haversine formula for distance calculation with database-level filtering using latitude/longitude bounds before applying the precise calculation to reduce computational overhead.

\subsection{Challenge 3: Blood Type Compatibility}

\textbf{Problem:} Emergency requests need to match not just exact blood types but also compatible types (e.g., O- is universal donor).

\textbf{Solution:} Created a \texttt{BLOOD\_COMPATIBILITY} constant dictionary in \texttt{utils/constants.py} defining compatible blood types, used in donor matching algorithms.

\section{Summary}

This chapter detailed the implementation of BDIMS, covering the development environment, technology stack (Django, PostgreSQL, Leaflet.js, Chart.js), project structure, and module-by-module implementation. Key implementations include the authentication system, donor management with eligibility checking, health metrics tracking, blood inventory management with automatic updates, emergency response with donor matching, and interactive location tracking. The chapter also presented important algorithms including eligibility calculation and distance measurement using the Haversine formula. Security measures including CSRF protection and role-based access control were implemented throughout. The next chapter covers the testing procedures used to validate the system.
