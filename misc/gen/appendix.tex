\appendix
\chapter{APPENDICES}

\section{User Manual}

\subsection{For Donors}

\subsubsection{Registration}

\textbf{Step 1:} Visit the BDIMS homepage

\textbf{Step 2:} Click on "Register as Donor" button

\textbf{Step 3:} Fill in the registration form:
\begin{itemize}
    \item Personal Information: Username, Email, Password, First Name, Last Name
    \item Blood Information: Blood Group
    \item Date of Birth (Must be 18-65 years old)
    \item Gender
    \item Contact: Phone Number, Address
    \item Health: Weight (minimum 50 kg)
\end{itemize}

\textbf{Step 4:} Click "Register" button

\textbf{Step 5:} You will be redirected to login page upon successful registration

\subsubsection{Login}

\textbf{Step 1:} Click "Login" on homepage

\textbf{Step 2:} Enter your username/email and password

\textbf{Step 3:} Click "Login" button

\textbf{Step 4:} You will be redirected to your donor dashboard

\subsubsection{Updating Profile}

\textbf{Step 1:} From dashboard, click "Update Profile"

\textbf{Step 2:} Update desired fields (phone, address, weight, medical conditions)

\textbf{Step 3:} Optionally upload a profile picture

\textbf{Step 4:} Click "Save Changes"

\subsubsection{Updating Location}

\textbf{Step 1:} Click "Update Location" from dashboard

\textbf{Step 2:} Use one of three methods:
\begin{itemize}
    \item Click "Use GPS" button to auto-detect your location
    \item Click on the map at your location
    \item Search for your address in the search box
\end{itemize}

\textbf{Step 3:} Verify the address shown is correct

\textbf{Step 4:} Click "Save Location"

\subsubsection{Adding Health Metrics}

\textbf{Step 1:} Click "Add Health Metrics" from dashboard

\textbf{Step 2:} Enter your current measurements:
\begin{itemize}
    \item Hemoglobin Level (g/dL)
    \item Blood Pressure: Systolic and Diastolic (mmHg)
    \item Weight (kg)
    \item Temperature (°C) [Optional]
    \item Pulse Rate (bpm) [Optional]
\end{itemize}

\textbf{Step 3:} Add any notes

\textbf{Step 4:} Click "Submit"

\textbf{Note:} Your eligibility status will be automatically updated based on these metrics.

\subsubsection{Scheduling a Donation}

\textbf{Step 1:} Click "Schedule Donation" from dashboard

\textbf{Step 2:} Check your eligibility status at the top

\textbf{Step 3:} If eligible, fill in:
\begin{itemize}
    \item Select Hospital/Blood Center
    \item Choose Date
    \item Select Preferred Time
    \item Add any notes
\end{itemize}

\textbf{Step 4:} Click "Submit Request"

\textbf{Step 5:} Wait for administrator approval (visible in dashboard)

\subsubsection{Viewing Donation History}

\textbf{Step 1:} Click "Donation History" from navigation

\textbf{Step 2:} View your complete donation records including:
\begin{itemize}
    \item Date and location of each donation
    \item Units donated
    \item Health parameters at time of donation
\end{itemize}

\subsubsection{Responding to Emergency Requests}

\textbf{Step 1:} Check your dashboard for emergency alerts (highlighted in red)

\textbf{Step 2:} Review emergency details (blood type needed, hospital, urgency)

\textbf{Step 3:} Check your eligibility status

\textbf{Step 4:} If eligible and willing, contact the hospital using provided contact information

\subsection{For Administrators}

\subsubsection{Accessing Admin Panel}

\textbf{Step 1:} Login with admin credentials

\textbf{Step 2:} You will be automatically directed to admin dashboard

\subsubsection{Managing Donors}

\textbf{View All Donors:}
\begin{itemize}
    \item Click "Manage Donors" from navigation
    \item Use search box to find specific donors
    \item Filter by blood group, eligibility, city
    \item Click on donor name to view detailed profile
\end{itemize}

\subsubsection{Approving Donation Requests}

\textbf{Step 1:} Click "Donation Requests" from navigation

\textbf{Step 2:} View list of pending requests

\textbf{Step 3:} Click "Review" on a request

\textbf{Step 4:} Check donor eligibility and details

\textbf{Step 5:} Choose action:
\begin{itemize}
    \item Click "Approve" to accept the request
    \item Click "Reject" and provide reason if declining
\end{itemize}

\subsubsection{Recording a Donation}

\textbf{Step 1:} Navigate to "Add Donation Record"

\textbf{Step 2:} Select the donor

\textbf{Step 3:} Fill in donation details:
\begin{itemize}
    \item Donation Date
    \item Hospital/Center
    \item Units Donated
    \item Pre-donation Health Metrics (BP, Hemoglobin, Weight)
\end{itemize}

\textbf{Step 4:} Add any notes

\textbf{Step 5:} Click "Save"

\textbf{Note:} Inventory will be automatically updated upon saving.

\subsubsection{Managing Blood Inventory}

\textbf{View Inventory:}
\begin{itemize}
    \item Click "Inventory Management"
    \item View current stock levels for all blood types
    \item Red indicates critical (<10 units)
    \item Orange indicates low (<25 units)
\end{itemize}

\textbf{Update Inventory:}
\begin{itemize}
    \item Click "Update" next to a blood type
    \item Enter new values for available and reserved units
    \item Add notes explaining the change
    \item Click "Save"
\end{itemize}

\subsubsection{Creating Emergency Requests}

\textbf{Step 1:} Click "Emergency Requests" → "Create New"

\textbf{Step 2:} Fill in emergency details:
\begin{itemize}
    \item Blood Group Required
    \item Units Needed
    \item Urgency Level (Critical/High/Medium)
    \item Hospital Location
    \item Patient Details
    \item Contact Information
\end{itemize}

\textbf{Step 3:} Click "Create Emergency Request"

\textbf{Step 4:} System will automatically match eligible donors

\textbf{Step 5:} View matched donors and their contact information

\subsubsection{Generating Reports}

\textbf{Step 1:} Click "Reports" from navigation

\textbf{Step 2:} Select report type:
\begin{itemize}
    \item Donor Statistics
    \item Donation History
    \item Inventory Status
    \item Emergency Requests Summary
\end{itemize}

\textbf{Step 3:} Select date range (if applicable)

\textbf{Step 4:} Click "Generate Report"

\textbf{Step 5:} View report on screen or export

\section{Installation Guide}

\subsection{Prerequisites}

\begin{itemize}
    \item Python 3.12 or higher
    \item PostgreSQL 13 or higher
    \item Git (for cloning repository)
    \item Modern web browser
\end{itemize}

\subsection{Installation Steps}

\subsubsection{Step 1: Clone Repository}

\begin{lstlisting}[language=bash]
git clone https://github.com/Codingincloud/BDIMS.git
cd BDIMS
\end{lstlisting}

\subsubsection{Step 2: Create Virtual Environment}

\textbf{Windows:}
\begin{lstlisting}[language=bash]
python -m venv venv
venv\Scripts\activate
\end{lstlisting}

\textbf{Linux/macOS:}
\begin{lstlisting}[language=bash]
python3 -m venv venv
source venv/bin/activate
\end{lstlisting}

\subsubsection{Step 3: Install Dependencies}

\begin{lstlisting}[language=bash]
pip install -r requirements.txt
\end{lstlisting}

\subsubsection{Step 4: Configure Database}

\textbf{Create PostgreSQL Database:}
\begin{lstlisting}[language=SQL]
CREATE DATABASE bdims_db;
CREATE USER bdims_user WITH PASSWORD 'your_secure_password';
ALTER ROLE bdims_user SET client_encoding TO 'utf8';
ALTER ROLE bdims_user SET default_transaction_isolation TO 'read committed';
ALTER ROLE bdims_user SET timezone TO 'UTC';
GRANT ALL PRIVILEGES ON DATABASE bdims_db TO bdims_user;
\end{lstlisting}

\textbf{Update settings.py:}
\begin{lstlisting}[language=Python]
DATABASES = {
    'default': {
        'ENGINE': 'django.db.backends.postgresql',
        'NAME': 'bdims_db',
        'USER': 'bdims_user',
        'PASSWORD': 'your_secure_password',
        'HOST': 'localhost',
        'PORT': '5432',
    }
}
\end{lstlisting}

\subsubsection{Step 5: Run Migrations}

\begin{lstlisting}[language=bash]
python manage.py makemigrations
python manage.py migrate
\end{lstlisting}

\subsubsection{Step 6: Create Superuser}

\begin{lstlisting}[language=bash]
python manage.py createsuperuser
\end{lstlisting}

Follow prompts to create admin account.

\subsubsection{Step 7: Load Sample Data (Optional)}

\begin{lstlisting}[language=bash]
python manage.py populate_hospitals
\end{lstlisting}

\subsubsection{Step 8: Collect Static Files}

\begin{lstlisting}[language=bash]
python manage.py collectstatic
\end{lstlisting}

\subsubsection{Step 9: Run Development Server}

\begin{lstlisting}[language=bash]
python manage.py runserver
\end{lstlisting}

\subsubsection{Step 10: Access Application}

Open browser and navigate to: \texttt{http://127.0.0.1:8000/}

\subsection{Production Deployment}

\textbf{For production deployment, additional steps required:}

\begin{enumerate}
    \item Set \texttt{DEBUG = False} in settings.py
    \item Configure \texttt{ALLOWED\_HOSTS} with your domain
    \item Set strong \texttt{SECRET\_KEY}
    \item Configure HTTPS with SSL certificate
    \item Use production WSGI server (Gunicorn or uWSGI)
    \item Set up reverse proxy (Nginx or Apache)
    \item Configure static file serving
    \item Set up automated backups
    \item Implement monitoring and logging
\end{enumerate}

\section{Code Samples}

\subsection{Eligibility Check Function}

\begin{lstlisting}[language=Python, caption=Complete Eligibility Verification]
def check_donor_eligibility(donor):
    """
    Comprehensive eligibility check for blood donation.
    Returns tuple: (is_eligible: bool, message: str, details: dict)
    """
    from datetime import date, timedelta
    from utils.constants import (
        MINIMUM_DONATION_INTERVAL_DAYS,
        HEMOGLOBIN_MIN_MALE,
        HEMOGLOBIN_MIN_FEMALE
    )
    
    details = {}
    
    # Age verification (18-65)
    age = donor.age
    if not (18 <= age <= 65):
        return False, f"Age must be between 18-65 (current: {age})", details
    details['age'] = {'status': 'pass', 'value': age}
    
    # Weight verification (>= 50 kg)
    if donor.weight < 50:
        return False, f"Weight must be >= 50 kg (current: {donor.weight} kg)", details
    details['weight'] = {'status': 'pass', 'value': float(donor.weight)}
    
    # Donation interval check (56 days)
    if donor.last_donation_date:
        days_since_donation = (date.today() - donor.last_donation_date).days
        if days_since_donation < MINIMUM_DONATION_INTERVAL_DAYS:
            days_remaining = MINIMUM_DONATION_INTERVAL_DAYS - days_since_donation
            next_eligible = date.today() + timedelta(days=days_remaining)
            return False, f"Must wait {days_remaining} more days (eligible on {next_eligible})", details
        details['interval'] = {'status': 'pass', 'days_since': days_since_donation}
    
    # Health metrics verification
    latest_metrics = donor.healthmetrics_set.first()
    if latest_metrics:
        # Hemoglobin check (gender-specific)
        min_hemoglobin = HEMOGLOBIN_MIN_MALE if donor.gender == 'M' else HEMOGLOBIN_MIN_FEMALE
        if latest_metrics.hemoglobin_level < min_hemoglobin:
            return False, f"Hemoglobin too low: {latest_metrics.hemoglobin_level} g/dL (min: {min_hemoglobin})", details
        details['hemoglobin'] = {
            'status': 'pass', 
            'value': float(latest_metrics.hemoglobin_level),
            'min_required': min_hemoglobin
        }
        
        # Blood pressure check
        if not (90 <= latest_metrics.blood_pressure_systolic <= 140):
            return False, "Blood pressure systolic out of range (90-140 mmHg)", details
        if not (60 <= latest_metrics.blood_pressure_diastolic <= 90):
            return False, "Blood pressure diastolic out of range (60-90 mmHg)", details
        details['blood_pressure'] = {
            'status': 'pass',
            'value': f"{latest_metrics.blood_pressure_systolic}/{latest_metrics.blood_pressure_diastolic}"
        }
    else:
        return False, "No health metrics recorded. Please add health metrics first.", details
    
    # All checks passed
    return True, "Eligible to donate blood", details
\end{lstlisting}

\subsection{Emergency Donor Matching Algorithm}

\begin{lstlisting}[language=Python, caption=Emergency Donor Matching with Location]
def find_matching_donors(emergency_request, max_distance_km=50):
    """
    Find eligible donors matching emergency blood request.
    Filters by blood type, eligibility, location, and preferences.
    """
    from donor.models import Donor
    from utils.constants import BLOOD_COMPATIBILITY
    
    # Get compatible blood types
    required_blood = emergency_request.blood_group
    compatible_types = BLOOD_COMPATIBILITY.get(required_blood, [required_blood])
    
    # Base query: matching blood type, eligible, allows emergency contact
    matching_donors = Donor.objects.filter(
        blood_group__in=compatible_types,
        is_eligible=True,
        allow_emergency_contact=True
    ).select_related('user')
    
    # Location-based filtering if hospital has coordinates
    if emergency_request.hospital.latitude and emergency_request.hospital.longitude:
        hospital_lat = float(emergency_request.hospital.latitude)
        hospital_lng = float(emergency_request.hospital.longitude)
        
        nearby_donors = []
        for donor in matching_donors:
            if donor.latitude and donor.longitude:
                distance = donor.distance_to(hospital_lat, hospital_lng)
                if distance and distance <= max_distance_km:
                    donor.calculated_distance = distance
                    nearby_donors.append(donor)
        
        # Sort by distance
        matching_donors = sorted(nearby_donors, key=lambda d: d.calculated_distance)
    
    # Additional eligibility verification
    verified_donors = []
    for donor in matching_donors:
        is_eligible, message, details = check_donor_eligibility(donor)
        if is_eligible:
            verified_donors.append(donor)
    
    return verified_donors
\end{lstlisting}

\section{System Screenshots}

\textit{Note: Screenshots would be included in the final printed report showing:}

\begin{enumerate}
    \item Homepage
    \item Donor Registration Form
    \item Donor Dashboard
    \item Interactive Location Update Map
    \item Health Metrics Form and History
    \item Donation Scheduling Interface
    \item Donation History View
    \item Admin Dashboard with Charts
    \item Donor Management (Admin)
    \item Inventory Management Interface
    \item Emergency Request Creation
    \item Blood Inventory Chart
\end{enumerate}

\section{Database Schema Diagram}

\textit{Note: A complete ER diagram would be included showing all entities, attributes, and relationships as described in Chapter 4.}

\section{Team Contribution}

\begin{table}[h]
\centering
\caption{Team Member Contributions}
\begin{tabular}{|p{3.5cm}|p{8cm}|}
\hline
\textbf{Team Member} & \textbf{Primary Responsibilities} \\
\hline
Bishal Shrestha (790310) & Backend Development (Models, Views), Database Design, Authentication Module, Testing \\
\hline
Chirayu Shrestha (790311) & Frontend Development (Templates, CSS), UI/UX Design, Interactive Maps, Responsive Design \\
\hline
Pappu Yadav (790324) & Inventory Management, Emergency System, Admin Panel, Charts Implementation \\
\hline
Prashant Ghimire (790328) & Documentation, Testing, Deployment, Presentation, Project Management \\
\hline
\end{tabular}
\end{table}

\textit{Note: All team members contributed to requirement analysis, system design, integration, and final testing. The above indicates primary focus areas.}

\section{Glossary}

\begin{description}
    \item[Blood Group] Classification of blood based on presence of antigens (A, B, AB, O) and Rh factor (+/-)
    \item[Donor] Individual who voluntarily gives blood
    \item[Eligibility] State of meeting all requirements to safely donate blood
    \item[Hemoglobin] Protein in red blood cells that carries oxygen; measured in g/dL
    \item[Inventory] Stock of available blood units by type
    \item[ORM] Object-Relational Mapping - technique to interact with database using objects
    \item[Unit] Standard measure of blood donation (typically 450-500 ml or 1 pint)
    \item[MVT] Model-View-Template - Django's architectural pattern
    \item[CSRF] Cross-Site Request Forgery - type of web security exploit
    \item[Haversine Formula] Mathematical formula to calculate great-circle distances between points on a sphere
\end{description}
